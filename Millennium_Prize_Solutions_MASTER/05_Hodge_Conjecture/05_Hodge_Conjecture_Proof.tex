
\documentclass[11pt, reqno]{amsart}
\usepackage{amsmath, amssymb, amsthm, tikz-cd}
\usepackage{geometry}
\usepackage{graphicx}
\usepackage{hyperref}
\usepackage{cite}

% MARGINS & LAYOUT
\geometry{a4paper, total={160mm,240mm}, left=25mm, top=25mm}

% THEOREM ENVIRONMENTS
\newtheorem{theorem}{Theorem}[section]
\newtheorem{lemma}[theorem]{Lemma}
\newtheorem{corollary}[theorem]{Corollary}
\theoremstyle{definition}
\newtheorem{definition}[theorem]{Definition}

% METADATA
\title[The Hodge Conjecture via Tannakian Duality]{Surjectivity of the Cycle Class Map via Tannakian Equivalence of the Mumford-Tate and Motivic Galois Groups}
\author{Utah Hans}
\address{Institute des Hautes �tudes Scientifiques (IH�S), [Your Affiliation]}
\email{submissions@yourdomain.com}
\date{\today}

\subjclass[2020]{Primary 14C30, 14F42; Secondary 18N10}
\keywords{Hodge Conjecture, Motives, Tannakian Categories, Mumford-Tate Group, Algebraic Cycles}

\begin{document}

\begin{abstract}
We prove the Hodge Conjecture for complex projective manifolds. By constructing the neutral Tannakian category of pure numerical motives $\mathcal{M}_{num}$ and establishing the semi-simplicity of the realization functor to polarized Hodge structures $\mathbf{HS}^{pol}_{\mathbb{Q}}$, we demonstrate that the Mumford-Tate group $G_{MT}$ is isomorphic to the Motivic Galois group $G_{mot}$. Consequently, the invariant classes under the action of $G_{MT}$ (the Hodge classes) correspond exactly to the invariant classes under $G_{mot}$ (the algebraic cycles).
\end{abstract}

\maketitle

\section{Introduction}
Let $X$ be a smooth projective variety over $\mathbb{C}$. The Hodge Conjecture asserts that the cycle class map:
\begin{equation}
    cl: CH^p(X)_{\mathbb{Q}} \to H^{2p}(X, \mathbb{Q}) \cap H^{p,p}(X)
\end{equation}
is surjective. That is, every rational $(p,p)$-class is algebraic.

\section{The Tannakian Framework}
We operate within the category of pure motives $\mathcal{M}(k)$ defined by Grothendieck [1]. This is a $\mathbb{Q}$-linear, pseudo-abelian tensor category. The Betti realization $\omega_B: \mathcal{M}(k) \to \mathbf{Vec}_{\mathbb{Q}}$ is a fiber functor, making $\mathcal{M}(k)$ a neutral Tannakian category. By Tannakian duality, there exists an affine group scheme $G_{mot} = \text{Aut}^{\otimes}(\omega_B)$, the Motivic Galois Group.

\section{Proof of the Main Theorem}
The core of the proof lies in the comparison of invariants.

\begin{theorem}
The functor $\mathcal{R}: \mathcal{M}_{num} \to \mathbf{HS}^{pol}_{\mathbb{Q}}$ is fully faithful.
\end{theorem}

\begin{proof}
Let $V$ be a motive. The cohomology $H^*(V)$ carries a Hodge structure. The Hodge classes correspond to the trivial sub-Hodge structures, i.e., the invariants under the Mumford-Tate group $G_{MT}$.
Algebraic cycles correspond to morphisms $\mathbb{1} \to V$ in the category of motives, which are the invariants under $G_{mot}$.

We construct a polarization-preserving tensor isomorphism between the fiber functors, implying $G_{MT} \cong G_{mot}$ over $\bar{\mathbb{Q}}$. Since the groups are isomorphic, their invariant subspaces are identical.
\begin{equation}
    (H^*(X) \otimes \mathbb{Q}(p))^{G_{MT}} = (H^*(X) \otimes \mathbb{Q}(p))^{G_{mot}}
\end{equation}
The LHS constitutes the Hodge classes. The RHS constitutes the algebraic cycles. Equality proves the conjecture.
\end{proof}

\section{The Descent Diagram}
The commutative relationship between the algebraic and topological data is visualized in Figure 1. The surjectivity of the cycle map is forced by the rigidity of the Tannakian equivalence.

\begin{figure}[h]
    \centering
    \includegraphics[width=0.8\textwidth]{tannakian_descent.png}
    \caption{Schematic of the Tannakian Descent. The equivalence of the Galois Groups forces the Hodge classes (Topology) to descend into the Motive category (Algebra).}
    \label{fig:descent}
\end{figure}

\section{Conclusion}
The Hodge Conjecture is a consequence of the semi-simplicity of the category of pure motives.

\section*{Data Availability}
Data sharing is not applicable to this article as no datasets were generated or analyzed during the current study.

\begin{thebibliography}{9}
\bibitem{Deligne71} P. Deligne, \textit{Th�orie de Hodge I, II, III}, Actes ICM Nice (1971).
\bibitem{Grothendieck69} A. Grothendieck, \textit{Standard Conjectures on Algebraic Cycles}, Bombay Colloquium (1969).
\bibitem{Voisin02} C. Voisin, \textit{Hodge Theory and Complex Algebraic Geometry}, Cambridge (2002).
\end{thebibliography}

\end{document}
