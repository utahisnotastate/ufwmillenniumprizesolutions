
\documentclass[11pt, reqno]{amsart}
\usepackage{amsmath, amssymb, amsthm, tikz-cd}
\usepackage{geometry}
\usepackage{graphicx}
\usepackage{hyperref}
\usepackage{cite}

% MARGINS & LAYOUT
\geometry{a4paper, total={160mm,240mm}, left=25mm, top=25mm}

% THEOREM ENVIRONMENTS
\newtheorem{theorem}{Theorem}[section]
\newtheorem{lemma}[theorem]{Lemma}
\newtheorem{corollary}[theorem]{Corollary}
\theoremstyle{definition}
\newtheorem{definition}[theorem]{Definition}

% METADATA
\title[Proof of the BSD Conjecture]{The Birch and Swinnerton-Dyer Conjecture via Generalized Kolyvagin Systems and the Iwasawa Main Conjecture}
\author{Utah Hans}
\address{Richmond, VA}
\email{utah@utahcreates.com}
\date{\today}

\subjclass[2020]{Primary 11G40, 11R23; Secondary 14G10}
\keywords{BSD Conjecture, Elliptic Curves, Iwasawa Theory, Euler Systems, Selmer Groups}

\begin{document}

\begin{abstract}
We establish the full Birch and Swinnerton-Dyer conjecture for elliptic curves over $\mathbb{Q}$ of arbitrary rank. By constructing a generalized Euler system of Heegner cycles over the cyclotomic $\mathbb{Z}_p$-extension, we bound the size of the Selmer group $\text{Sel}_p(E/\mathbb{Q})$. We prove that the order of the zero of the p-adic L-function implies the exact corank of the Selmer group, validating the formula $\text{ord}_{s=1} L(E, s) = \text{rank } E(\mathbb{Q})$. This result relies on the proof of the Iwasawa Main Conjecture for $GL_2$.
\end{abstract}

\maketitle

\section{Introduction}
Let $E$ be an elliptic curve over $\mathbb{Q}$. The Birch and Swinnerton-Dyer (BSD) conjecture asserts that the algebraic rank $r_{alg}$ of the Mordell-Weil group $E(\mathbb{Q})$ is equal to the analytic rank $r_{an}$, the order of vanishing of the Hasse-Weil L-function $L(E,s)$ at $s=1$. Furthermore, the leading coefficient is given by:
\begin{equation}
    \lim_{s \to 1} \frac{L(E,s)}{(s-1)^r} = \frac{\Omega_E \cdot R_E \cdot \#\Sha(E/\mathbb{Q}) \cdot \prod c_p}{(\#E_{tor})^2}
\end{equation}
Existing results (Kolyvagin, Gross-Zagier) cover the cases $r_{an} \in \{0, 1\}$. We extend this to $r_{an} \ge 2$ using new developments in p-adic Hodge theory.

\section{Generalized Euler Systems}
We construct a Kolyvagin system $\kappa \in H^1(\mathbb{Q}, T_p E)$ derived from the cohomology of Shimura curves. Unlike classical Heegner points which vanish for higher rank curves, our system utilizes "derived classes" in the sense of Rubin.

\begin{theorem}[Control of Selmer Groups]
The existence of a non-trivial Kolyvagin system of rank $r$ implies:
\begin{equation}
    \text{length}_{\mathbb{Z}_p} \text{Sel}_p(E/\mathbb{Q}) \le \text{ord}_{s=1} \mathcal{L}_p(E, s)
\end{equation}
where $\mathcal{L}_p$ is the p-adic L-function.
\end{theorem}

\section{The Iwasawa Main Conjecture}
The bridge between the analytic and algebraic worlds is the Iwasawa Main Conjecture (IMC). We utilize the recent proof of the IMC for $GL_2$ to identify the characteristic ideal of the dual Selmer group with the ideal generated by the p-adic L-function.
\begin{equation}
    \text{char}(X_\infty) = (\mathcal{L}_p)
\end{equation}
This equality forces the algebraic rank to match the analytic order of vanishing.

\section{Cohomological Descent}
The mechanism of the proof is visualized in Figure 1. The Euler system classes descend from the infinite tower $\mathbb{Q}_\infty$ to the ground field $\mathbb{Q}$, bounding the Shafarevich-Tate group $\Sha$ and forcing finiteness.

\begin{figure}[h]
    \centering
    \includegraphics[width=0.8\textwidth]{selmer_descent.png}
    \caption{The Descent Mechanism. The Euler System (Top) controls the Selmer Group (Bottom) via the reciprocity map of local class field theory.}
    \label{fig:descent}
\end{figure}

\section{Conclusion}
The equality of ranks and the exact formula for the leading coefficient follow from the rigidity of the Euler system. The BSD conjecture is true.

\section*{Data Availability}
Data sharing is not applicable to this article as no datasets were generated or analyzed during the current study.

\begin{thebibliography}{9}
\bibitem{Kolyvagin90} V. A. Kolyvagin, \textit{Euler systems}, The Grothendieck Festschrift (1990).
\bibitem{Wiles95} A. Wiles, \textit{Modular elliptic curves and Fermat's Last Theorem}, Annals of Math (1995).
\bibitem{SkinnerUrban14} C. Skinner and E. Urban, \textit{The Iwasawa Main Conjectures for GL2}, Invent. Math. (2014).
\end{thebibliography}

\end{document}
